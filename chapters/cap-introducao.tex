%% ------------------------------------------------------------------------- %%n
\chapter{Introdução}
\label{cap:introducao}

\section{Contexto}
\label{sec:intro_contexto}

O estudo da planctologia é de grande importância na comunidade científica, principalmente na oceanografia. O nome plâncton vem do Grego planktos e significa errante, que vaga ou flutua. Caracteriza-se, assim, por organismos planctônicos os que não possuem o poder de locomoção suficiente para evitar o transporte passivo pelas massas de água [\cite{calazans2011organismos}].  A importância do estudo dessas criaturas se dá não somente por serem responsáveis por cerca de 45\% do oxigênio produzido mundialmente [\cite{brierleyplankton}],  através da fotossíntese do fitoplâncton, mas também pela grande diversidade de classes e de características como forma, tamanho e até da natureza do local de coleta [\cite{calazans2011organismos}]. 


Existem diversas maneiras de se fazer a amostragem de plânctons como, por exemplo, através de garrafas, redes, bombas de sucção e sistemas ópticos [\cite{calazans2011organismos}]. A amostragem de criaturas marítimas é datada desde de 1829, quando Thompson utilizou uma rede para coletar larvas de crustáceos e de cracas [\cite{brierleyplankton}]. Mas foi Victor Hensen, em 1887, o primeiro pesquisador a desenvolver, de fato, um sistema para coleta de amostragens de plânctons de forma quantitativa [\cite{benfield2007rapid, wiebe2003hensen, allen1919contribution}]. Naquela época estavam interessados em responder três questões fundamentais: Quais organismos planctônicos estão presentes no mar?  Quantos de cada tipo estão presentes? Como a composição dos plânctons muda com o passar do tempo?  [\cite{benfield2007rapid}]. Essas questões continuam atuais e graças ao avanço de sistemas e equipamentos de coleta, assim como de técnicas computacionais, há um grande esforço científico para a sua evolução. 


\section{Problema}
\label{sec:intro_problema}

Após a fase de coleta, especificamente através de sistemas ópticos, tem-se um conjunto de imagens de plânctons que não possuem rótulos. O grande desafio desse é que demanda-se muito tempo e pessoas especializadas para cumprir essa tarefa, já que elas possuem, por exemplo, uma grande variedade de classes, formas e tamanhos. Como resultado, normalmente temos uma lacuna muito grande entre o tempo de coleta das imagens e a rotulação que seja base para um classificador. Somado a isso, muitas vezes, já temos um novo equipamento que possuem fotos com características diferentes das anteriores. Ou seja, apesar de todo o avanço que tivemos com equipamentos e técnicas, esse ainda é um problema que ocorre desde 1800 [\cite{benfield2007rapid}]. Torna-se necessário, dessa forma, encontrar uma maneira de reduzir o esforço gasto em rotulação de amostras.  
    
O objetivo do aprendizado computacional é que tenhamos, no final do processo, um classificador que consiga determinar a qual classe alguma nova imagem um plâncton pertence. Isso só é possível quando temos uma base de treinamento. Para tanto é necessário que, dado as imagens coletadas, alguém as rotule. 


\section{Proposta}
\label{sec:intro_proposta}

Aprendizado Computacional é o estudo de sistemas que melhoram através da experiência. Dentro deste contexto existe, na literatura, algumas formas de trabalhar com nenhum ou pouco dados rotulados. Temos duas principais vertentes: Aprendizado Ativo e Aprendizado Semi-Supervisionado [\cite{zhu2006semi, abu2012learning}].  Este trabalho pretende mostrar que através de técnicas de aprendizado ativo conseguimos obter um bom classificador através 


\section{Trabalhos Relacionados}
\label{sec:intro_relacionados}

Texto texto texto texto texto texto texto texto texto texto texto texto texto texto texto texto texto texto texto texto texto texto texto texto texto texto


\section{Organização}
\label{sec:intro_organizacao}

Texto texto texto texto texto texto texto texto texto texto texto texto texto texto texto texto texto texto texto texto texto texto texto texto texto texto